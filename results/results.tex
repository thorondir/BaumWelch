\documentclass{article}
\usepackage[a4paper, total={6in, 8in}]{geometry}
\usepackage{tikz}
\usepackage{pgfplots}
\usepackage{listings}
\usepackage{multicol}
\usepackage[style=numeric]{biblatex}
\addbibresource{bib.bib}


\begin{document}
Training set with 13 observations, 2-17 states, 100 iterations (no random $\pi$???)
\begin{figure}[h]\centering
\begin{tikzpicture}
  \begin{axis}[xtick={0,50,100,150,200,256}]
    \addplot[color=red] table[col sep=comma, x index=0, y index=1] {100its_test_set.csv};
  \end{axis}
\end{tikzpicture}
\caption{LogLikelihood of models run with $n$ numbers of states}\label{lap}
\end{figure}
Training set of first 10 (shuffled), 82 events, 76 unique logs
\begin{figure}[h]\centering
\begin{tikzpicture}
  \begin{axis}[legend style = {font = \footnotesize, legend pos = south west}, xtick = {0,16,32,48,64,82,96,112,128}]
    \addplot[color=red] table[col sep=comma, x index=0, y index=1] {10traces_test_set.csv};
    \addlegendentry{Test Set}
    \addplot[color=blue] table[col sep=comma, x index=0, y index=1] {10traces_10traces.csv};
    \addlegendentry{10 Training Traces}
    \addplot[color=black] coordinates {(82,-160)(82,0)};
    %\addplot[color=black] coordinates {(76,-160)(76,0)};
  \end{axis}
\end{tikzpicture}
\caption{LogLikelihood of models run with $n$ numbers of states, when trained with a small training set}\label{lap}
\end{figure}

%\end{multicols}
\printbibliography
\end{document}